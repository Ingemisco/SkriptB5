\documentclass[
twoside=semi,
fontsize=12,
DIV=12, 
cleardoublepage=current,
leqno,
headings=optiontoheadandtoc, 
toc=idx
]{scrbook}

\usepackage[german]{babel}
\usepackage[utf8]{inputenc}
\usepackage[T1]{fontenc}
\usepackage{color}
\usepackage{datetime}
\usepackage{cancel}
\usepackage{tikz-cd}
\usepackage{tcolorbox}
\usepackage{tkz-euclide}

\usepackage[headsepline]{scrlayer-scrpage}
\setkomafont{pageheadfoot}{\normalfont}
\setkomafont{pagefoot}{\slshape}

\defpagestyle{main}
{
	{\thesection\ \leftmark \hfill}{\hfill \thesection\ \leftmark}{}
}
{
	{\pagemark\hfill}{\hfill \pagemark}{}
}

\defpagestyle{sectionstart}
{
	{}{}{}(0pt,0pt)
}
{
	{\pagemark\hfill}{\hfill \pagemark}{}
}

\usepackage{chngcntr}\counterwithout{equation}{section}

\usepackage{amsmath}
\usepackage{amssymb}
\usepackage{amsthm}
\usepackage{stmaryrd}
\usepackage{enumerate}
\usepackage[pass]{geometry}
\usepackage{csquotes}
\usepackage{mathtools}
\usepackage{nicematrix}
\usepackage[shortlabels]{enumitem}
\usepackage{hyperref}
\MakeOuterQuote{"}
\usepackage{romanbar}

\newcommand{\N}{\mathbb{N}}
\newcommand{\Z}{\mathbb{Z}}
\newcommand{\Q}{\mathbb{Q}}
\newcommand{\C}{\mathbb{C}}
\newcommand{\R}{\mathbb{R}}
\newcommand{\F}{\mathbb{F}}
\renewcommand{\P}{\mathbb{P}}
\newcommand{\A}{\mathbb{A}}
\newcommand{\V}{\mathcal{V}}
\newcommand{\nsquare}{\cancel{\square}}
\newcommand{\x}{\underline{x}}
\newcommand{\I}{\mathfrak{I}}

\newcommand{\brac}[1]{\left( #1 \right)}
\newcommand{\bracB}[1]{\left[ #1 \right]}
\newcommand{\bracC}[1]{\left< #1 \right>}
\newcommand{\abs}[1]{\left| #1 \right|}
\newcommand{\set}[1]{\left\{ #1 \right\}}
\newcommand{\legendre}[2]{\brac{\frac{#1}{#2}}}

\newcommand{\emphasize}[1]{\underline{#1}}


\newcommand{\textcase}[2]{$\begin{Bmatrix} \textrm{#1} \\ \textrm{#2}\end{Bmatrix}$}
\newcommand{\case}[2]{\begin{Bmatrix} #1 \\ #2\end{Bmatrix}}
\newcommand{\schemaSmith}[3]{\begin{NiceTabular}[corners={NE,SE}, hvlines]{cc} #1 & #2 \\ #3 \end{NiceTabular}}
\newcommand{\schemaSmithH}[2]{\begin{NiceTabular}[corners={NE,SE}, hvlines]{cc} #1 & #2 \end{NiceTabular}}
\newcommand{\schemaSmithV}[2]{\begin{NiceTabular}[corners={NE,SE}, hvlines]{c} #1 \\ #2 \end{NiceTabular}}

\renewcommand{\thechapter}{\Roman{chapter}}

\newcommand*{\ORIGchapterheadstartvskip}{}
\let\ORIGchapterheadstartvskip=\chapterheadstartvskip
\renewcommand*{\chapterheadstartvskip}{
	\ORIGchapterheadstartvskip
	\noindent\rule[\baselineskip]{\linewidth}{4pt}\par
}
\newcommand*{\ORIGchapterheadendvskip}{}
\let\ORIGchapterheadendvskip=\chapterheadendvskip
\renewcommand*{\chapterheadendvskip}{
	\ORIGchapterheadendvskip
	\noindent\rule[\baselineskip]{\linewidth}{4pt}\par
}



\DeclareMathOperator{\im}{im}
\DeclareMathOperator{\supp}{supp}
\DeclareMathOperator{\ann}{ann}
\DeclareMathOperator{\id}{id}
\DeclareMathOperator{\rk}{rk}
\DeclareMathOperator{\End}{End}
\DeclareMathOperator{\Aut}{Aut}
\DeclareMathOperator{\Hom}{Hom}
\DeclareMathOperator{\GL}{GL}
\DeclareMathOperator{\com}{com}
\DeclareMathOperator{\qf}{qf}
\DeclareMathOperator{\irr}{irr}
\DeclareMathOperator{\tr}{tr}
\DeclareMathOperator{\Char}{char}
\DeclareMathOperator{\Nil}{Nil}
\DeclareMathOperator{\Spec}{Spec}

\swapnumbers
\theoremstyle{definition}
\newtheorem{definition}{Definition}[section]
\newtheorem{bemerkung}[definition]{Bemerkung}
\newtheorem{beispiel}[definition]{Beispiel}
\newtheorem{warnung}[definition]{Warnung}
\newtheorem{satz}[definition]{Satz}
\newtheorem{lemma}[definition]{Lemma}
\newtheorem{proposition}[definition]{Proposition}
\newtheorem{notation}[definition]{Notation}
\newtheorem{korollar}[definition]{Korollar}
\newtheorem{erinnerung}[definition]{Erinnerung}
\newtheorem{erinnerung-und-sprechweise}[definition]{Erinnerung und Sprechweise}
\newtheorem{prop-def}[definition]{Proposition und Definition}
\newtheorem{def-prop}[definition]{Definition und Proposition}
\newtheorem{def-satz}[definition]{Definition und Satz}
\newtheorem{def-ueb}[definition]{Definition und \"Ubung}
\newtheorem{satz-not}[definition]{Satz und Notation}
\newtheorem{satz-def}[definition]{Satz und Definition}
\newtheorem{uebung}[definition]{\"Ubung}
\newtheorem{motivation}[definition]{Motivation}
\newtheorem{bem-not}[definition]{Bemerkung und Notation}
\newtheorem{sprechweise}[definition]{Sprechweise}
\newtheorem{defueb}[definition]{Definition und \"Ubung}
\newtheorem{theorem}[definition]{Theorem}

\begin{document}
	\tableofcontents\thispagestyle{empty}
	\newpage\thispagestyle{empty}
	\mainmatter
	\chapter{Affine Algebraische Variet\"aten}
	\pagestyle{main}
	
	\section{Algebraische Erg\"anzungen}
	Ringe sind kommutativ (mit $1$), Ringhomomorphismen $\varphi: A\to B$ erf\"ullen $\varphi(1) = 1$. Erzeugte Ideale 
		\[\bracC{a_1, \dots, a_n} = \sum_{i=1}^{n} Aa_i = \set{\sum_{i=1}^{n} a_ib_i \mid b_1, \dots, b_n \in A}\]
	
	\begin{definition}\label{1.1.1}\hfill\newline
		Sei $A$ ein Ring, sei $I \subseteq A$ ein Ideal
		\begin{enumerate}[(a)]
			\item $\sqrt{I} := \set{a\in A \mid \exists n \geq 1 \textrm{ mit } a^n \in I}$ ist ein Ideal in $A$, genannt das \emphasize{Radikal} von $I$.
			
			\medskip\noindent
			Ist $I = \sqrt{I}$, so hei\ss t $I$ ein \emphasize{Radikalideal}.
			
			\item $\Nil(A) := \sqrt{\set{0}}$ hei\ss t das \emphasize{Nilradikal} von $A$. Die Elemente von $\Nil(A)$ hei\ss en die \emphasize{nilpotenten} Elemente von $A$.
			
			\item Der Ring $A$ hei\ss t \emphasize{reduziert}, wenn $\Nil(A) = \set{0}$ ist
		\end{enumerate}
	\end{definition}
	
	\begin{proof}
		\begin{enumerate}[(a)]
			\item Seien $a, b \in \sqrt{I}$, etwa $a^m \in I, b^n \in I$. Dann
				\[(a+b)^{m+n} = \sum_{i+j=m+n} \binom{m+n}{i} \underbrace{a^ib^j}_{\in I} \in I  \implies a+b \in \sqrt{I}\]
			Und $(ac)^m = a^mc^m \in I$ f\"ur $c \in A \implies ac \in \sqrt{I}$
 		\end{enumerate}
	\end{proof}

	\begin{bemerkung}\label{1.1.2}\hfill
		\begin{itemize}
			\item $I \subseteq \sqrt{I}$
			\item Jedes Primideal ist ein Radikalideal.
			\item $A= \Z, n = p_1^{e_1} \cdots p_r^{e_r}$ mit $p_i$ prim, $e_i \geq 1$, dann ist 
			$\sqrt{\bracC{n}} = \bracC{p_1 \cdots p_r}$
		\end{itemize}
	\end{bemerkung}

	\begin{lemma}\label{1.1.3}\hfill\newline
		Seien $I, I_1, I_2 \subseteq A$ Ideale.
		\begin{enumerate}[(a)]
			\item $\sqrt{I_1 \cap I_2} = \sqrt{I_1} \cap \sqrt{I_2}$
			
			\item $\sqrt{I} = \bracC{1} \iff I = \bracC{1}$
			
			\item Ist $J \supseteq I$ ein weiteres Ideal, dann ist $\sqrt{J/I} = \sqrt{J}/I$ (in $A/I$).
			
			\medskip\noindent
			Insbesondere: Der Ring $A/I$ ist reduziert genau dann, wenn $I = \sqrt{I}$
		\end{enumerate}
	\end{lemma}
	
	\begin{proof}
		Aufgabe 4.
	\end{proof}

	\begin{satz}\label{1.1.4}\hfill\newline
		F\"ur jedes Ideal $I \subseteq A$ ist 
			\[\sqrt{I} = \bigcap_{\substack{\mathfrak{p} \in \Spec(A)\\\mathfrak{p} \supseteq I}} \mathfrak{p}\]
		Die Radikalideale sind also die Durchschnitte von Primidealen.
	\end{satz}
	
	\begin{proof}\hfil\newline
		"$\subseteq$": Aus $I \subseteq \mathfrak{p}$ und $\mathfrak{p}$ prim folgt $\sqrt{I} \subseteq \sqrt{\mathfrak{p}} = \mathfrak{p}$
		
		\medskip\noindent
		"$\supseteq$": Sei $t \in A, t \notin \sqrt{I}$, sei $A_t := A_S$ mit $S:= \set{1,t,t^2, \dots}$.\newline
		Sei $\varphi:A \to A_t, \varphi(a) = \frac{a}{1}$. Dann ist $IA_t := I_t := \set{\frac{a}{t^n} \mid a \in I, n \geq 0}$ ein Ideal in $A_t$ und $1 \notin IA_t$ [Angenommen, $1 = \frac{1}{1} = \frac{a}{t^n}$ mit $a \in I \implies \exists m: t^m(t^n-a) = 0$, d.h. $t^{m+n} = at^m \in I \lightning$].\newline
		Also existiert ein maximales Ideal $\mathfrak{m}$ von $A_t$ mit $IA_t \subseteq \mathfrak{m} \implies \mathfrak{p} := \varphi^{-1}(\mathfrak{m}) \in \Spec(A)$ mit $I \subseteq \mathfrak{p}$ und $t \notin \mathfrak{p}$.
	\end{proof}
	
	\begin{korollar}\label{1.1.5}\hfill\newline
		$\Nil(A) = \displaystyle \bigcap_{\mathfrak{p}\in \Spec(A)} \mathfrak{p}$
	\end{korollar}

	\begin{definition}\label{1.1.6}\hfill\newline
		Sei $R$ ein Ring
		\begin{enumerate}[(a)]
			\item Eine \emphasize{$R$-Algebra} ist ein Ringhomomorphismus $\varphi:R \to A$ (in einen Ring $A$). Sprechweise "$A$ ist eine $R$-Algebra".
			
			\item Seien $\alpha:R \to A, \beta:R\to B$ zwei $R$-Algebren. Ein \emphasize{Homomorphismus von $R$-Algebren} (oder $R$-Homomorphismus) von $A$ nach $B$ ist ein Ringhomomorphismus $\varphi:A \to B$ mit $\varphi \circ \alpha = \beta$.
			
 			\item $R$-Algebren $A, B$ hei\ss en \emphasize{isomorph} (als \emphasize{$R$-Algebren}), wenn es einen bijektiven \newline $R$-Homomorphismus $\varphi:A \to B$ gibt.
		\end{enumerate}
	\end{definition}

	\begin{beispiel}\label{1.1.7}\hfill
		\begin{enumerate}[1.]
			\item Sei $R \subseteq A$ Teilring, dann ist $A$ eine $R$-Algebra via Inklusion.
			
			\item \label{1.1.7.2} Sei $A$ eine $R$-Algebra, dann gibt es eine Bijektion
			\begin{align*}
				\Hom(R[x_1, \dots, x_n], A) &\to A\times \cdots \times A = A^n\\
				\varphi &\mapsto (\varphi(x_1), \dots, \varphi(x_n))
			\end{align*}
		
			\item Zwei $R$-Algebren k\"onnen als Ring isomorph sein, ohne es als $R$-Algebren zu sein (Aufgabe 2).
		\end{enumerate}
	\end{beispiel}

	\begin{lemma}\label{1.1.8}\hfill\newline
		F\"ur jede $R$-Algebra $\alpha:R \to A$ sind \"aquivalent:
		\begin{enumerate}[(i)]
			\item\label{1.1.8.1} $\exists n \in \N: \exists a_1, \dots, a_n \in A$, so dass $A$ als Ring von $\alpha(R)$ und $a_1, \dots, a_n$ erzeugt wird.
			
			\item\label{1.1.8.2} $\exists n \in \N: \exists$ surjektiver Homomorphismus $R[x_1, \dots, x_n] \to A$ von $R$-Algebren.
		\end{enumerate}
		Gelten (i) und (ii), so hei\ss t die $R$-Algebra $A$ \emphasize{endlich erzeugt}.
	\end{lemma}

	\begin{proof}
		Einfach (verwende \hyperref[1.1.7.2]{1.7.2})
	\end{proof}

	\begin{bemerkung}\label{1.1.9}\hfill\newline
		Ist die $R$-Algebra $A$ als $R$-Modul endlich erzeugt, so auch als $R$-Algebra. Die Umkehrung ist im Allgemeinen falsch (Beispiel $A~=~R[x]$).
	\end{bemerkung}

	\begin{definition}\label{1.1.10}\hfill\newline
		Sei $\varphi: A \to B$ ein $A$-Algebra
		\begin{enumerate}[(a)]
			\item $b \in B$ hei\ss t \emphasize{ganz} \"uber $A$, falls es eine Identit\"at $b^n+\varphi(a_1)b^{n-1} + \cdots + \varphi(a_n) = 0$ gibt ($n \in \N, a_i \in A$).
			
			\item Die $A$-Algebra $B$ hei\ss t \emphasize{ganz}, wenn jedes $b \in B$ ganz \"uber $A$ ist.\newline
			Leicht zu sehen: $b \in B$ ist ganz \"uber $A$ genau dann, wenn $A[b]$ ($:=$ der von $\varphi(A)$ und $b$ erzeugte Teilring von $B$) als $A$-Modul endlich erzeugt ist.
		\end{enumerate}
	\end{definition}

	\begin{satz}\label{1.1.11}\hfill\newline
		Sei $B$ eine $A$-Algebra. Dann ist $C:= \set{b \in B\mid b \textrm{ ganz \"uber } A }$ ein Teilring von $B$, genannt der \emphasize{ganze Abschluss} von $A$ in $B$.
	\end{satz}

	\begin{definition}\label{1.1.12}\hfill\newline
		Ein Ring hei\ss t \emphasize{noethersch}, wenn jedes Ideal von $A$ endlich erzeugt ist. \newline
		Beispiel: Hauptidealringe sind noethersch. Ist $A$ noethersch, so auch $A/I$ und $A_S$ ($I$ Ideal, $S \subseteq A$ multiplikative Menge).
	\end{definition}

	\begin{theorem}\label{1.1.13} Hilbertscher Basissatz\newline
		Ist $A$ noethersch, so auch $A[x]$.
	\end{theorem}

	\begin{korollar}\label{1.1.14}\hfill\newline
		Ist $A$ noethersch, so ist auch jede endlich erzeugte $A$-Algebra noethersch.
	\end{korollar}
	
	\begin{proof}
		\hyperref[1.1.13]{1.13} und \hyperref[1.1.8.2]{1.8(ii)}
	\end{proof}

	\newpage
	\section{Affine Algebraische Mengen}\thispagestyle{sectionstart}
	$k$: (fixierter) Grundk\"orper, $\overline{k}$: (ein) algebraischer Abschluss von $k$. Sei $k \subseteq K$ eine (beliebige) K\"orpererweiterung mit $K$ algebraisch abgeschlossen (z.B. $K=\overline{k}$). \newline
	Sei $n \in \N, \x = (x_1, \dots, x_n), k[\x] = k[x_1, \dots, x_n]$.
	
	\begin{definition}\label{1.2.1}\hfill
		\begin{enumerate}[(a)]
			\item $\A^n := K^n$ der $n$-dimensionale affine Raum
			
			\item Sei $P \subseteq k[\x]$ eine Menge von Polynomen, dann schreibe 	
				\[\V(P) := \V_{\A^n}(P) := \set{\xi \in K^n \mid \forall p \in P: p(\xi) = 0}\]
			Variet\"at von $P$.
			\item Eine Teilmenge $V \subseteq \A^n$ hei\ss t \emphasize{affine $k$-Variet\"at}, falls $\exists P \subseteq k[\x]$ mit $V = \V(P)$.
			
			\item Sind $W \subseteq V \subseteq \A^n$ affine $k$-Variet\"aten, so hei\ss t $W$ eine \emphasize{$k$-Untervariet\"at} von $V$. 
		\end{enumerate}
	\end{definition}

	\begin{lemma}\label{1.2.2}\hfill\newline
		Sei $P \subseteq k[\x]$ eine Menge, sei $I:= \bracC{P}$ (das von $P$ erzeugte Ideal in $k[\x]$). Dann ist 
		$\V(P) = \V(\sqrt{I})$.
	\end{lemma}

	\begin{proof}\hfill\newline
		"$\supseteq$": klar
		
		\medskip\noindent
		"$\subseteq$": Sei $\xi \in \V(P)$, sei $f \in \sqrt{I}$, also $f^m = g_1p_1 + \cdots +g_rp_r$ mit $m \in \N, g_i \in k[\x], p_i \in P$. Auswerten in $\xi$ liefert:
			\[f(\xi)^m = \sum_{i=1}^r g_i(\xi)\underbrace{p_i(\xi)}_{=0} = 0\]
		Also $f(\xi) = 0$.
	\end{proof}

	\begin{korollar}\label{1.2.3}\hfill\newline
		Jede affine $k$-Variet\"at hat die Form $V=\V(f_1, \dots, f_r)$ mit endlich 
		vielen $f_1, \dots, f_r \in k[\x]$
	\end{korollar}

	\begin{proof}\hfill\newline
		Sei $V = \V(P)$ mit einer Teilmenge $P \subseteq k[\x]$. Sei $\bracC{P} = \bracC{f_1, \dots, f_r}$ mit $f_i \in k[\x]$ (\hyperref[1.1.13]{Hilberscher Basissatz}). Dann ist $V = \V(f_1, \dots, f_r)$ mit \hyperref[1.2.2]{Lemma 2.2}
	\end{proof}

	\begin{lemma}\label{1.2.4}\hfill
		\begin{enumerate}[(a)]
			\item $\emptyset, \A^n$ sind affine $k$-Variet\"aten.
			
			\item Sind $V_1, V_2 \subseteq \A^n$ affine $k$-Variet\"aten, so auch $V_1 \cup V_2$.
			
			\item Sind $V_\lambda \subseteq \A^n$ affine $k$-Variet\"aten f\"ur $\lambda \in \Lambda$, so auch $\displaystyle \bigcap_{\lambda \in \Lambda} V_\lambda$.
			
			\item Sind $V \subseteq \A^m, W \subseteq \A^n$ affine $k$-Variet\"aten, so auch $V\times W \subseteq \A^m \times \A^n = \A^{m+n}$.
		\end{enumerate}
	\end{lemma}

	\begin{proof}\hfill
		\begin{enumerate}[(a)]
			\item $\emptyset = \V(1), \A = \V(0)$
			
			\item Seien $V_i = \V(I_i)$ mit $I_i \subseteq k[\x]$ Ideale ($i = 1, 2$).
			\begin{tcolorbox}[colback=white,colframe=blue,arc=0cm]
				\textbf{Behauptung}: $V_1 \cup V_2 = \V(I_1 \cap I_2)$.
				
				\tcblower
				\textbf{Begr\"undung}: \newline
				"$\subseteq$": klar
				
				\medskip\noindent
				"$\supseteq$": Sei $\xi \in \V(I_1 \cap I_2)$. Ohne Einschr\"ankung $\xi \notin V_1$, also gibt es $f \in I_1$ mit $f(\xi) \neq 0$. F\"ur jedes $g \in I_2$ ist $fg \in I_1 \cap I_2$, also nach Voraussetzung ist $\underbrace{f(\xi)}_{\neq 0}g(\xi) = 0$. Also ist $g(\xi) = 0$, also $\xi \in \V(I_2) = V_2$.
			\end{tcolorbox}
			
			\item \[\bigcap_{\lambda \in \Lambda} \V(I_\lambda) = \V\brac{\bigcup_{\lambda \in \Lambda} I_\lambda}\]
			
			\item Sei $V = \V_{\A^m}(I)$ mit $I \subseteq k[x_1, \dots, x_m]$. Sei $W = \V_{\A^n}(J)$ mit $J \subseteq k[y_1, \dots, y_n]$. Dann ist
			$V\times W = \V_{\A^m\times \A^n}(I \cup J)$
		\end{enumerate}
	\end{proof}

	\begin{korollar}\label{1.2.5}\hfill\newline
		Sind $I_1, I_2, I_\lambda$ ($\lambda \in \Lambda$) Ideale in $k[\x]$, so ist
		\begin{enumerate}[(a)]
			\item $\V(I_1)\cup \V(I_2) = \V(I_1V_2) = \V(I_1 \cap I_2)$
			
			\item $\displaystyle \bigcap_{\lambda \in \Lambda} \V(I_\lambda) = \V\brac{\sum_{\lambda \in \Lambda} I_\lambda}$
		\end{enumerate}
	\end{korollar}

	\begin{definition}\label{1.2.6}\hfill\newline
		Ist $\V \subseteq \A^n$ eine Teilmenge, so hei\ss t $\I(V) = \I_k(V) := \set{f \in k[\x] \mid \forall \xi \in V: f(\xi) = 0}$ das \emphasize{(Verschwindungs-)Ideal} von $V$.
	\end{definition}

	\begin{lemma}\label{1.2.7}\hfill\newline
		Seien $V, V_1, V_2 \subseteq \A^n$ Teilmengen
		
		\begin{enumerate}[(a)]
			\item $\I_k(V)$ ist ein Radikalideal in $k[\x]$. Ist $V$ eine affine $k$-Variet\"at, so ist $\V(\I_k(V)) = V$.
			
			\item Aus $V_1 \subseteq V_2$ folgt $\I_k(V_2) \subseteq \I_k(V_1)$. Ist $V_2$ eine affine $k$-Variet\"at, so gilt auch $\I_k(V_2) \subseteq \I_k(V_1) \implies V_1 \subseteq V_2$.
			
			\item $\I_k(V_1 \cup V_2) = \I_k(V_1) \cap \I_k(V_2)$.
			
			\item Jede absteigende Folge $V_1 \supseteq V_2 \supseteq \cdots$ von affinen $k$-Variet\"aten in $\A^n$ wird station\"ar.
		\end{enumerate}
	\end{lemma}

	\begin{proof}\hfill
		\begin{enumerate}[(a)]
			\item $\I_k(V)$ ist ein Radikalideal: klar.
			
			\medskip\noindent
			Sei $V = \V(I)$ eine affine $k$-Variet\"at mit $I$ Ideal. \newline Dann ist $I \subseteq \I_k(V)$, also $\V(\I_k(V)) \subseteq \V(I) = V$. Umgekehrt ist $V \subseteq \V(\I_k(V))$ trivial.
			
			\item "$\implies$" ist trivial.
			
			\medskip\noindent
			"$\impliedby$" (falls $V$ eine affine $k$-Variet\"at ist) folgt aus
				\[V_1 \subseteq \V(\I(V_1)) \subseteq \V(\I(V_2)) \overset{(a)}{=} V_2.\]
			
			\item offensichtlich.
			
			\item $\I(V_1) \subseteq \I(V_2) \subseteq \cdots$ wird station\"ar, weil $k[\x]$ noethersch ist. Ist $\I(V_i) = \I(V_{i+1})$, so auch $V_i = V_{i+1}$ nach (b).
		\end{enumerate}
	\end{proof}

	\begin{beispiel}\label{1.2.8}\hfill
		\begin{enumerate}[1.]
			
			\item $\I(\emptyset) = \bracC{1} = k[\x]$
			
			\smallskip\noindent
			$\I(\A^n) = \set{0}$ (Ist $0 \neq f \in k[\x]$, so existiert $\xi \in K^n$ mit $f(\xi) \neq 0$, wegen $|K| = \infty$)
			
			\item Einige Beispiele:\newline
			\begin{center}
				\begin{tabular}{l|c} % TODO: Add images
					$f$ & $\V(f)$\\\hline
					$x_1^2+x_2^2-1$ & \\
					$x_1x_2$ & \\
					$x_2^2-x_1^2-x_1^3$ & \\
					$x_2^2-x_1^3$ & \\
					$x_1^2+x_2^2$ & 
				\end{tabular}
				
				\medskip
				In $\A^3$:\newline
				\begin{tabular}{l|c} % TODO: Add images
					$f$ & $\V(f)$\\\hline
					$x_1^2+x_2^2-1$ & \\
					$x_1^2+x_2^2-x_3^2 -1$ & \\
					$x_1^2-x_2^2x_3$ & 
				\end{tabular}
			\end{center}
			
			
			\item Ist $V = \V(f)$ mit $f \in k[\x] \setminus k$, so hei\ss t $V$ eine Hyperfl\"ache in $\A^n$.
			
			\item Ist $V = \V_{\A^n}(f_1, \dots, f_r)$ mit $\deg(f_i) = 1$, so hei\ss t (ist) $V$ ein affin-linearer Teilraum des $\A^n$.
			
			\item Endliche Teilmengen von $k^n$ sind affine $k$-Variet\"aten in $\A^n$: Ist $\xi = (\xi_1, \dots, \xi_n) \in k^n$, so ist
			$\set{\xi} = \V(x_1-\xi_1, \dots, x_n-\xi_n)$.
			
			\item Rational parametrisierte Variet\"aten:
			\begin{align*}
				\V(x_1^2-x_2^3 &= \set{(t^2,t^3) \mid t \in \A^n})\\
				\V(x_1^2-x_2^2-1) &= \set{\brac{\frac{t^2+1}{t^2-1}, \frac{2t}{t^2-1}} \mid t \in \A^n \setminus \set{\pm 1}} \cup \set{(1,0)}
			\end{align*}
		\end{enumerate}
	\end{beispiel}

	\begin{theorem}\label{1.2.9} (Hilbertscher Nullstellensatz, k\"orpertheoretische Form)\newline
		Sei $k \subseteq F$ eine K\"orpererweiterung. Ist $F$ als $k$-Algebra endlich erzeugt, so ist $k \subseteq F$ endlich algebraisch.
	\end{theorem}
	
	\begin{proof}\hfill\newline
		Es gibt $0\neq \alpha_1, \dots, \alpha_n \in F$ mit $F = k[\alpha_1, \dots, \alpha_n]$. Durch Induktion nach $n$ zeigen wir $F/k$ ist algebraisch.
		
		\noindent $n = 1$: Es gibt $p \in k[\alpha_1]$ mit $\frac{1}{\alpha_1} = p(\alpha_1)$, also ist $\alpha_1$ algebraisch \"uber $k$.
		
		\medskip\noindent
		$n-1 \to n$: Es ist auch $F = k(\alpha_1)[\alpha_2, \dots, \alpha_n]$. Durch Induktion sind $\alpha_2, \dots, \alpha_n$ algebraisch \"uber $k(\alpha_1)$. Es gen\"ugt daher zu zeigen,
		dass $\alpha_1$ algebraisch \"uber $k$ ist. Es gibt Gleichungen
			\[u_i\alpha_i^{d_i} + \sum_{j=1}^{d_i-1} v_{ij}\alpha_i^j = 0\]
		in $F$ ($i = 1, \dots, n$) mit $u_i, v_{ij} \in k[\alpha_1], u_i \neq 0$.
		Dividiere durch $u_i$, daraus folgt, dass $\alpha_i$ ganz \"uber $k[\alpha_1, \frac{1}{u_i}] \subseteq k(\alpha_1)$ ist ($i = 2, \dots, n$).
		F\"ur $u:= u_2 \cdots u_n$ gilt also $0 \neq u \in k[\alpha_1]$ und $\alpha_2, \dots, \alpha_n$ sind ganz \"uber $k[\alpha_1, \frac{1}{u}]$. 
		Also ist $k[\alpha_1, \frac{1}{u}] \subseteq F$ eine ganze Ringerweiterung.
		Angenommen $\alpha_i$ ist transzendent \"uber $k$. W\"ahle $f \in k[\alpha_1]$ irreduzibel mit $f \nmid u$. Es gibt eine Ganzheitsgleichung f\"ur $\frac{1}{f}$: $\brac{\frac{1}{f}}^m + b_1\brac{\frac{1}{f}}^{m-1}+\cdots + b_m = 0$ in $F$ mit $b_1, \dots b_m \in k[\alpha_1, \frac{1}{u}]$. Multiplizieren mit $f^m$ und einer hohen Potenz von $u$ liefert eine Gleichung $u^N+c_1f+\cdots+c_mf^m = 0$ mit $N \geq 0, c_1,\dots, c_m \in k[\alpha_1]$. Widerspruch zu $f\nmid u$ und $f$ irreduzibel.
	\end{proof}

	\begin{korollar}\label{1.2.10}\hfill\newline
		Ist $k$ ein K\"orper, $A$ eine endlich erzeugte $k$-Algebra. Dann ist f\"ur jedes maximale Ideal $\mathfrak{m}$ von $A$ $A/\mathfrak{m}$ eine \emphasize{endliche} K\"orpererweiterung von $k$.
	\end{korollar}

	\begin{proof}\hfill\newline
		$A/\mathfrak{m}$ ist eine endlich erzeugte $k$-Algebra (und K\"orpererweiterung), also $[A/\mathfrak{m}:k] < \infty$ nach \hyperref[1.2.9]{2.9}.
	\end{proof}

	\begin{korollar}\label{1.2.11}(Hilbertscher Nullstellensatz, geometrische Form)\newline
		F\"ur jedes Ideal $I \neq \bracC{1}$ in $k[\x]$ ist $\V(I) \neq \emptyset$.
	\end{korollar}

	\begin{proof}\hfill\newline
		Es gibt ein maximales Ideal $\mathfrak{m} \subseteq k[\x]$ mit $I \subseteq \mathfrak{m}$, also $[\frac{k[\x]}{\mathfrak{m}}:k] < \infty$ nach Korollar \hyperref[1.2.10]{2.10}.
		Es gibt eine $k$-Einbettung $\varphi: k[\x]/\mathfrak{m} \to K$. Setze $\xi_i := \varphi(x_i+\mathfrak{m}) \in K$ ($i=1,\dots, n$), $\xi:= (\xi_1, \dots, \xi_n)$.
		\begin{tcolorbox}[colback=white,colframe=blue,arc=0cm]
			\textbf{Behauptung}: $\xi \in \V(I)$.
			
			\tcblower
			\textbf{Begr\"undung}: \newline
			F\"ur $f \in k[\x]$ ist $\varphi(f + \mathfrak{m}) = f(\xi)$, denn $f \in I \implies f \in \mathfrak{m} \implies f(\xi) = 0$.
		\end{tcolorbox}
	\end{proof}

	\begin{bemerkung}\label{1.2.12}\hfill\newline
		Haben $f_1, \dots, f_r \in k[\x]$ \emphasize{keine} gemeinsame Nullstelle in $\A^n$, so existieren $g_1, \dots, g_r \in k[\x]$ mit $f_1g_1+\cdots f_rg_r = 1$.
	\end{bemerkung}

	\begin{theorem}\label{1.2.13} Hilbertscher Nullstellensatz, Idealtheoretische Version\newline
		F\"ur jedes Ideal $I \subseteq k[\x]$ ist $\I(\V(I)) = \sqrt{I}$.
	\end{theorem}
	
	\noindent
	NB: Aus \hyperref[1.2.13]{2.13} folgt sofort wieder \hyperref[1.2.11]{2.11}
	
	\begin{proof}
		$I \subseteq \I(\V(I))$ ist klar, also auch $\sqrt{I} \subseteq \I(\V(I))$. Umgekehrt sei $0 \neq f \in \I(\V(I))$. Sei $J:= $ das von $I$ und von $tf-1$ erzeugte Ideal in $k[\x, t] = k[x_1, \dots, x_n, t]$. Es ist $\V_{\A^{n+1}}(J) \neq \emptyset$, denn sei $(\xi, \tau) \in \A^n\times \A^1$ mit $(\xi, \tau) \in \V(J)$, dann $\xi \in \V(I)$ und $\tau \underbrace{f(\xi)}_{=0} = 1$, Widerspruch. Also $J =\bracC{1}$ in $k[\x, t]$. 
		\newline 
		Nach \hyperref[1.2.11]{2.11} haben wir eine Identit\"at $1 = p \cdot (tf-1) + \sum_{i=1}^{r}q_if_i$ mit $r\geq 0, p, q_1, \dots, q_r \in k[\x, t], f_1, \dots, f_r \in I$. Substituiere dann $t = \frac{1}{f}$, d.h. betrachte den $k[\x]$-Homomorphismus $\varphi: k[\x, t] \to k[\x, \frac{1}{f}] \subseteq k(\xi)$ mit $\varphi(t) = \frac{1}{f}$. Das gibt 
		\begin{align}
			1 &= \sum_{i=1}^r\varphi(q_i)f_i \tag{$*$}\label{1.2.13.1}
		\end{align}
		in $k[\x, \frac{1}{f}]$. Dabei ist $\varphi(q_i) = \frac{g_i}{f^{e_i}}$ mit $e_i \geq 0, f_i \in k[\x]$. Multipliziere (\ref{1.2.13.1}) mit einer hohen Potenz von $f$. Es folgt $f^e \in \bracC{f_1, \dots, f_r} \subseteq I$ (in $k[\x]$), also $f \in \sqrt{I}$.
	\end{proof}

	\begin{beispiel}\label{1.2.14}\hfill\newline
		Sei $V = \V(f)$ eine Hyperfl\"ache, $f \in k[\x]\setminus k$. Dann ist $\I(V) = \bracC{g}$ mit $\bracC{g} = \sqrt{\bracC{f}}$, $g$ ist das Produkt der \emphasize{verschiedenen}
		irreduziblen Faktoren von $f$.
	\end{beispiel}

	\begin{korollar}\label{1.2.15}\hfill\newline
		F\"ur Ideale $I, J \subseteq k[\x]$ gilt $\V(I) \subseteq \V(J) \iff \sqrt{I} \supseteq \sqrt{J}$.
	\end{korollar}

	\begin{proof}
		$\V(I) \subseteq \V(J) \iff \I(\V(I)) \supseteq \I(\V(J)) \overset{\hyperref[1.2.13]{2.13}}{\iff} \sqrt{I} \supseteq \sqrt{J}$
	\end{proof}

	\begin{korollar}\label{1.2.16}\hfill\newline
		\begin{align*}
			\begin{Bmatrix}
				\textrm{affine } k\textrm{-Variet\"aten }\\V \textrm{in } \A^n
			\end{Bmatrix}
			&\to \begin{Bmatrix}
				\textrm{Ideale } I \subseteq k[\x]\\
				\textrm{mit } \sqrt{I} = I
			\end{Bmatrix}\\
			V &\mapsto \I(V)
		\end{align*}
	
		ist eine ($\subseteq$-umkehrende) Bijektion mit Umkehrabbildung $\V(I) \mapsfrom I$
	\end{korollar}

	\begin{proof}
		\hyperref[1.2.7]{2.7} und \hyperref[1.2.13]{2.13}.
	\end{proof}

	\begin{korollar}\label{1.2.17}\hfill
		\begin{enumerate}[(a)]
			\item Seien $V_\lambda \subseteq \A^n$ ($\lambda \in \Lambda$) affine $k$-Variet\"aten, dann ist 
				\[\I\brac{\bigcap_{\lambda \in \Lambda} V_{\lambda}} = \sqrt{\sum_{\lambda \in \Lambda} \I(V_\lambda)}.\]
			
			\item Seien $V \subseteq \A^m, W \subseteq \A^n$ affine $k$-Variet\"aten, dann ist $\I(V \times W) = \sqrt{I}$ mit $I:= $ das von $\I(V) \subseteq k[\x]$ und $\I(W) \subseteq k[\underline{y}]$ erzeugte Ideal in $k[x_1, \dots, x_n, y_1, \dots, y_n]$.
		\end{enumerate}
	\end{korollar}

	\newpage
	\section{Die Zariskitopologie}\thispagestyle{sectionstart}
	\begin{definition}\label{1.3.1}\hfill
		\begin{enumerate}[(a)]
			\item Die $k$-\emphasize{Zariskitopologie} auf $\A^n$ hat genau die affinen $k$-Variet\"aten in $\A^n$ als abgeschlossene Mengen. Diese Mengen nennen wir auch die $k$-abgeschlossenen Mengen.
			
			\item F\"ur eine Teilmenge $X \subseteq \A^n$ definieren wir die $k$-Zariskitopologie auf $X$ als die Relativtopologie .
		\end{enumerate}
	\end{definition}

	\noindent Bemerkung: Vergleiche Lemma \hyperref[1.2.4]{2.4}(a)-(c). Die offenen $k$-Variet\"aten in $\A^n$ sind stabil unter endlicher Vereinigung und beliebigem Schnitt (und enth\"alt $\emptyset$).
	
	\begin{bemerkung}\label{1.3.2}
		\begin{enumerate}[1.]
			\item 
			F\"ur $X \subseteq \A^n$ sei $\overline{X}:=$ der Abschluss von $X$ in der $k$-Zariskitopologie. Es ist $\overline{X} = \V(\I(X))$, denn "$\subseteq$" ist klar und
			$\V(\I(X)) \subseteq \V(\I(\overline{X})) \overset{\hyperref[1.2.7]{2.7(a)}}{=} \overline{X}$
			
			\item
			F\"ur $f \in k[\x]$ ist $D(f) = D_{\A^n}(f) = \set{\xi \in \A^n\mid f(\xi) \neq 0} = \A^n \setminus \V(f)$ eine $k$-offene Menge. Jede $k$-offene Menge hat die Form $D(f_1) \cup \cdots \cup D(f_r)$ mit $f_i \in k[\x]$   
			
			\item Die Zariskitopologie ist " sehr nicht-Hausdorff", z.B. $k = \C = K$. Die abgeschlossenen Mengen in $\A^1$ sind nur $\A^1$ sowie alle endlichen Mengen.
			
			\item Die Zariskitopologie auf $\A^n\times \A^n$ ist \emphasize{nicht} die Produkttopologie! (Aufgabe 7)
			
			\item Ist $k \subseteq E \subseteq K$ ein Zwischenk\"orper, so ist die $E$-Zariskitopologie im allgemeinen strikt feiner als die $k$-Zariskitopologie.
		\end{enumerate}
	\end{bemerkung}

	\begin{definition}\label{1.3.3}\hfill\newline
		Sei $X$ ein topologischer Raum.
		\begin{enumerate}[(a)]
			\item $X$ hei\ss t \emphasize{irreduzibel}, falls $X \neq \emptyset$ und gilt: Sind $X_1, X_2 \subseteq X$ abgeschlossen und $X = X_1 \cup X_2$, so ist $X_1 = X$ oder $X_2 = X$. Andernfalls hei\ss t $X$ \emphasize{reduzibel}.
			
			\item $Y \subseteq X$ hei\ss t eine irreduzible Komponente von $X$, falls $Y$ irreduzibel, aber jede echte Obermenge von $Y$ in $X$ reduzibel ist. 
		\end{enumerate}
	\end{definition}

	\begin{lemma}\label{1.3.4}\hfill\newline
		Sei $X \neq \emptyset$ ein topologischer Raum.
		\begin{enumerate}[(a)]
			\item Es sind \"aquivalent
			\begin{enumerate}[(i)]
				\item $X$ ist irreduzibel
				\item jede offene Menge $\neq \emptyset$ in $X$ ist dicht in $X$.
 				\item Je zwei nichtleere offenen Mengen in $X$ haben nichtleeren Schnitt.
			\end{enumerate}
			
			\item F\"ur $Y \subseteq X$ gilt $Y$ ist irreduzibel genau dann, wenn $\overline{Y}$ irreduzibel ist.
			
			\item Die irreduziblen Komponenten von $X$ sind abgeschlossen
		\end{enumerate}
	\end{lemma}

	\begin{bemerkung}\label{1.3.5}\hfill
		\begin{enumerate}[1.]
			\item $\A^n$ ist irreduzibel: F\"ur $f, g\in k[\x] \setminus k$ ist $D(f) \cap D(g) = D(fg) \neq \emptyset$.
			\item Ein Hausdorffraum $X$ ist irreduzibel genau dann, wenn $|X| = 1$.
		\end{enumerate}
	\end{bemerkung}

	\begin{lemma}\label{1.3.6}\hfill\newline
		Sei $X$ ein topologischer Raum. Jede irreduzible Teilmenge von $X$ ist in einer irreduziblen Komponente von $X$ enthalten. Insbesondere ist $X$ Vereinigung seiner irreduziblen Komponenten.
	\end{lemma}

	\begin{proof}
		Sei $Y \subseteq X$ irreduzibel. Betrachte $\mathcal{M} := \set{Z \subseteq X \mid Z \textrm{ irreduzibel, } Z \supseteq Y}$. Ist $(Z_i)_{i \in I}$ eine bez\"uglich Inklusion total geordnete Familie in $\mathcal{M}$, so ist $Z:= \bigcup_{i \in I}Z_i$ irreduzibel, also $Z \in \mathcal{M}$: Benutze \hyperref[1.3.4]{3.4(a)(iii)}. Seien $U_1, U_2 \subseteq X$ offen mit $U_j \cap Z \neq \emptyset$ ($j = 1, 2$). Dann gibt es $i_1, i_2 \in I$ mit $U_1 \cap Z_{i_1} \neq \emptyset$ und $U_2 \cap Z_{i_2} \neq \emptyset$. Wir k\"onnen erreichen, dass $i_1 = i_2 =:i$. Wegen $Z_i$ irreduzibel ist $U_1 \cap U_2 \cap Z_i \neq \emptyset$.
	\end{proof}

	\begin{definition}\label{1.3.7}\hfill\newline
		Ein topologischer Raum $X$ hei\ss t \emphasize{noethersch}, wenn jede absteigende Folge $Z_1 \supseteq Z_2 \supseteq \cdots $ von abgeschlossenen Teilmengen von $X$ station\"ar wird.
	\end{definition}

	\begin{lemma}\label{1.3.8}\hfill\newline
		Es sind \"aquivalent:
		\begin{enumerate}[(i)]
			\item $X$ ist noethersch.
			
			\item Jedes System $\neq \emptyset$ aus abgeschlossenen Teilmengen hat ein minimales Element.
			
			\item Jede offene Menge in $X$ ist quasikompakt
		\end{enumerate}
	\end{lemma}

	\begin{proof}
		(i)$\iff$(ii) ist klar, (i) $\iff$ jede aufsteigende Folge offener Mengen wird station\"ar - Das ist (iii).
	\end{proof}

	\begin{bemerkung}\label{1.3.9}\hfill
		\begin{enumerate}[1.]
			\item Jede affine $k$-Variet\"at ist ein noetherscher topologischer Raum.
			
			\item Jeder Teilraum eines noetherschen Raumes ist noethersch.
		\end{enumerate}
	\end{bemerkung}

	\begin{satz}\label{1.3.10}\hfill\newline
		Ein noetherscher topologischer Raum hat nur endlich viele irreduzible Komponenten.
	\end{satz}

	\begin{proof}
		Sei $\mathcal{M}$ die Menge aller abgeschlossenen Mengen $\emptyset \neq Y \subseteq X$, die \emphasize{nicht} Vereinigung von endlich vielen irreduziblen sind. Zu zeigen ist $\mathcal M = \emptyset$.
		Angenommen, $\mathcal{M} \neq \emptyset$. Nach \hyperref[1.3.8]{3.8(ii)} existiert ein minimales $Y \in \mathcal{M}$. $Y$ ist reduzibel, etwa $Y = Y_1 \cup Y_2$ mit abgeschlossenen $Y_i \subsetneq Y$ ($i = 1, 2$). Also $Y_1, Y_2 \notin \mathcal{M}$, also $Y \notin \mathcal{M} \lightning$. Also $\mathcal{M} = \emptyset$, das hei\ss t, es gibt irreduzible abgeschlossene $X_1, \dots, X_r \subseteq X$ mit $X = X_1 \cup \cdots \cup X_r$. Ohne Einschr\"ankung $X_i \nsubseteq X_j$ f\"ur $i \neq j$. Dann sind $X_1, \dots, X_r$ die irreduziblen Komponenten von $X$. Denn sei $Y \subseteq X$ eine irreduzible Teilmenge von $X$. $Y = (Y\cap X_1) \cup \cdots\cup (Y \cap X_r)$, also $Y \cap X_i = Y$ f\"ur ein $i$, also $Y = X_i$.
	\end{proof}

	\begin{bemerkung}
		-
	\end{bemerkung}

	\begin{korollar}\label{1.3.12}\hfill\newline
		Sei $V \subseteq \A^n$ eine affine $k$-Variet\"at. Dann gibt es endlich viele irreduzible $k$-Variet\"aten $V_1, \dots, V_r \subseteq \A^n$ mit $V = V_1 \cup \dots \cup V_r$ und $V_i \nsubseteq V_j$ f\"ur $i\neq j$. Dadurch sind $V_1, \dots, V_r$ eindeutig bestimmt (bis auf Reihenfolge) und sind die $k$-irreduziblen Komponenten von $V$.
	\end{korollar}

	\begin{satz}\label{1.3.13}\hfill\newline
		Sei $V$ eine affine $k$-Variet\"at. Dann ist $V$ genau dann $k$-irreduzibel, wenn $\I_k(V) $ ein Primideal in $k[\x]$ ist. 
	\end{satz}

	\begin{proof}
		Inhalt...
	\end{proof}
\end{document}
